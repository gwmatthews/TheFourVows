%    sutras for tracing, reading, chanting

\documentclass[letterpaper]{article}
\usepackage[margin=0.5cm]{geometry}

\usepackage{sutras}

%%% CHOICE OF FONTS
%\setCJKmainfont{KanjiStrokeOrders}  %% for tracing 
\setCJKmainfont{ipam.ttf}  %% serif
%\setCJKmainfont{ipaexm.ttf}  %% serif
%\setCJKmainfont{ipag.ttf}  %% sans
%\setCJKmainfont{ipamp.ttf}  %% sans
%\setCJKmainfont{ipaexg.ttf}  %% sans
%\setCJKmainfont{YOzM90.ttf}  %% calligraphic
%\setCJKmainfont{Harano Aji Mincho SemiBold}  %% calligraphic
%\setCJKmainfont{KouzanGyoushoOTF.otf}  %% calligraphic

%%%%%%%
%%%%%% END PREAMBLE
%%%%%%%

\begin{document}
	\vspace*{3cm}
	
	\centering {\Huge 聖経 
		
		\vspace*{0.5cm}
		
		\href{https://github.com/gwmatthews/TheFourVows}{\texttt{sutras.sty}}
		
		{\small for stylish sutras}
	}


	

	\vspace*{2cm}
	    
	    \begin{itemize}
	    	\item[]  Some \LaTeX code for typesetting kanji texts.
	    	\item[] Format for tracing, reading (with or without meanings and readings), and display.
	    	\item[] \cc George Matthews, 2020
	    \end{itemize}
	    
	    
	    
	
	\vspace*{1cm}
	
	\begin{quotation}
		
	\begin{verbatim}
	
	\usepackage{sutras}
	
	\begin{multicols}{4}
	\RLmulticolcolumns
	
	  \heartblock{r}{K}{m}               % reading above; kanji large gray; meaning below
	
	  \heartflow{K}                      % dark gray text, variable boxes for kanji
	
	  \heartspaced{K}                    % variable boxes with additional padding
	
	  \haiku{abc \coda}                  % two columns \coda = 。
	
	  \compactverse{abcdef \stick ...}   % tightly set verse \stick = vertical 一
	
	  \spacedverse{abcdef \stick  ...}   % spaciously set verse
	
	\end{multicols}
	\end{verbatim}
	
\end{quotation}
	
	\vspace*{1cm}
 
	\begin{itemize}
		
			\item[] {\Large 
			 \href{https://github.com/gwmatthews/TheFourVows}{Source code is here.}}
			 \item[] {\Large 
			 	\href{https://gwmatthews.github.io/the-four-vows.pdf}{Demo is here.}}
		
	\end{itemize}
	
	
	\vfill\eject\pagebreak
	
	\vspace*{2cm}
	
	{\LARGE Examples}
	
	\vspace*{1cm}
	
	
	
	\begin{itemize}
		\item[] {\LARGE For tracing}
		\begin{itemize}
			\item[] \href{https://gwmatthews.github.io/the-four-vows-stroke-order.pdf}{KanjiStrokeOrders.ttf}
		\end{itemize}
	\end{itemize}
	
	\vspace*{1cm}
	
	\begin{itemize}
		\item[] {\LARGE Serif}
		\begin{itemize}
			\item[] \href{https://gwmatthews.github.io/the-four-vows-ipaexm.pdf}{ipaexm.ttf}
			\item[] \href{https://gwmatthews.github.io/the-four-vows-ipamp.pdf}{ipamp.ttf}
			\item[] 
			\href{https://gwmatthews.github.io/the-four-vows-ipam.pdf}{ipam.ttf}
		\end{itemize}
	\end{itemize}
	
	\vspace*{1cm}
	
	\begin{itemize}
		\item[] {\LARGE Sans Serif}
		\begin{itemize}
			\item[] \href{https://gwmatthews.github.io/the-four-vows-ipaexg.pdf}{ipaexg.ttf}
			\item[] 
			\href{https://gwmatthews.github.io/the-four-vows-ipag.pdf}{ipag.ttf}
		\end{itemize}
	\end{itemize}

\vspace*{1cm}

\begin{itemize}
	\item[] {\LARGE Calligraphic}
	\begin{itemize}
		\item[] \href{https://gwmatthews.github.io/the-four-vows-HaranoAjiMinchoSB.pdf}{Harano Aji Mincho SemiBold}
		\item[] \href{https://gwmatthews.github.io/the-four-vows-YOzM90.pdf}{YOzM90.ttf}
		\item[] \href{https://gwmatthews.github.io/the-four-vows-KouzanGyousho.pdf}{KouzanGyoushoOTF.otf}
	\end{itemize}
\end{itemize}
\vspace*{1cm}

\begin{itemize}
\item[] {\LARGE Verse}
\begin{itemize}
	\item[] \href{https://gwmatthews.github.io/verse-test.pdf}{verse styles}
	
\end{itemize}

\end{itemize}
	
	
	
	
	\vfill\eject\pagebreak
	
\centering 四弘誓願 The Four Bodhisattva Vows

\begin{multicols}{4}
\RLmulticolcolumns
\renewcommand{\kanji}{\centering\fontsize{45}{45}}

\heartblock{しゅう shu}{衆}{many}
\heartblock{じょう jou}{生}{life \\ 衆生 all living things}
\heartblock{む mu}{無}{nothing}
\heartblock{へん hen}{邊}{boundary \\ 邊 is now 辺,  無辺 limitless}
\heartblock{せい sei}{誓}{vow}
\heartblock{がん gan}{願}{vow  \\ 誓願 vow}
\heartblock{ど do}{度}{degrees \\ paramita, save}

\heartblock{ぽん hon}{煩}{anxiety}
\heartblock{のう nou}{悩}{trouble \\ 煩悩 three poisons }
\heartblock{む mu}{無}{no}
\heartblock{じん jin}{儘}{exhaust \\ simplifies to 盡,  侭,  尽}
\heartblock{せい sei}{誓}{}
\heartblock{がん gan}{願}{}
\heartblock{だん dan}{斷}{quit \\ simplifies to 断}

\heartblock{ほう hou}{法}{dharma}
\heartblock{もん mon}{門}{gate}
\heartblock{む mu}{無}{no}
\heartblock{りょう ryo}{量}{quantity \\ 無量 immeasurable}
\heartblock{せい sei}{誓}{}
\heartblock{がん gan}{願}{}
\heartblock{がく gaku}{學}{learn \\ simplifies to 学}

\heartblock{ぶつ butsu}{佛}{Buddha}
\heartblock{どう dou}{道}{path}
\heartblock{む mu}{無}{no}
\heartblock{じょう jou}{上}{above \\ 無上 best}
\heartblock{せい sei}{誓}{}
\heartblock{がん gan}{願}{}
\heartblock{じょう jou}{成}{become}

\end{multicols}

\begin{verbatim}

    \heartblock{}{}{}

\end{verbatim}

\pagebreak

\renewcommand{\kanji}{\centering\fontsize{60}{60}}

\vspace*{0.5cm}
\begin{multicols}{4}
	\RLmulticolcolumns
	\heartblock{}{衆}{}
	\heartblock{}{生}{}
	\heartblock{}{無}{}
	\heartblock{}{邊}{}
	\heartblock{}{誓}{}
	\heartblock{}{願}{}
	\heartblock{}{度}{}
	
	\columnbreak
	
	\heartblock{}{煩}{}
	\heartblock{}{悩}{}
	\heartblock{}{無}{}
	\heartblock{}{儘}{}
	\heartblock{}{誓}{}
	\heartblock{}{願}{}
	\heartblock{}{斷}{}
	
	\columnbreak
	
	\heartblock{}{法}{}
	\heartblock{}{門}{}
	\heartblock{}{無}{}
	\heartblock{}{量}{}
	\heartblock{}{誓}{}
	\heartblock{}{願}{}
	\heartblock{}{學}{}
	
	\columnbreak
	
	\heartblock{}{佛}{}
	\heartblock{}{道}{}
	\heartblock{}{無}{}
	\heartblock{}{上}{}
	\heartblock{}{誓}{}
	\heartblock{}{願}{}
	\heartblock{}{成}{}
	
\end{multicols}

\begin{verbatim}

    \heartblock{}{無}{}
    
\end{verbatim}

\vfill\eject\pagebreak

%%%%% MEDIUM FLOWING TEXT

\vspace*{1cm}

\renewcommand{\kanji}{\centering\fontsize{55}{55}}
\vspace*{2cm}

\begin{multicols}{4}
	\RLmulticolcolumns
	
	\heartflow{衆}
	\heartflow{生}
	\heartflow{無}
	\heartflow{邊}
	\heartflow{誓}
	\heartflow{願}
	\heartflow{度}
	\columnbreak
		
	\heartflow{煩}
	\heartflow{悩}
	\heartflow{無}
	\heartflow{儘}
	\heartflow{誓}
	\heartflow{願}
	\heartflow{斷}
	\columnbreak

	\heartflow{法}
	\heartflow{門}
	\heartflow{無}
	\heartflow{量}
	\heartflow{誓}
	\heartflow{願}
	\heartflow{學}
	\columnbreak
	
	\heartflow{佛}
	\heartflow{道}
	\heartflow{無}
	\heartflow{上}
	\heartflow{誓}
	\heartflow{願}
	\heartflow{成}

\end{multicols}

\begin{verbatim}

    \heartflow{}
    
\end{verbatim}

\vfill\eject\pagebreak

%%%%% SMALL FLOWING TEXT
\vspace*{2cm}
\renewcommand{\kanji}{\centering\fontsize{40}{40}}
\textcolor{gray}{\kanji{四弘誓願}}
\vspace*{2cm}

\renewcommand{\kanji}{\centering\fontsize{35}{35}}
\begin{multicols}{7}
	
	\RLmulticolcolumns
	%%% 1
	\heartflow{衆}
	\heartflow{生}
	\heartflow{無}
	\heartflow{邊}
	\heartflow{誓}
	\heartflow{願}
	\heartflow{度}
	%%% 2
	\heartflow{煩}
	\heartflow{悩}
	\heartflow{無}
	\heartflow{儘}
	\heartflow{誓}
	\heartflow{願}
	\heartflow{斷}
	%%% 3
	\heartflow{法}
	\heartflow{門}
	\heartflow{無}
	\heartflow{量}
	\heartflow{誓}
	\heartflow{願}
	\heartflow{學}
	%%%4
	\heartflow{佛}
	\heartflow{道}
	\heartflow{無}
	\heartflow{上}
	\heartflow{誓}
	\heartflow{願}
	\heartflow{成}
    \coda
	%%%
	%%% 1
	\heartflow{衆}
	\heartflow{生}
	\heartflow{無}
	\heartflow{邊}
	\heartflow{誓}
	\heartflow{願}
	\heartflow{度}
	%%% 2
	\heartflow{煩}
	\heartflow{悩}
	\heartflow{無}
	\heartflow{儘}
	\heartflow{誓}
	\heartflow{願}
	\heartflow{斷}
	%%% 3
	\heartflow{法}
	\heartflow{門}
	\heartflow{無}
	\heartflow{量}
	\heartflow{誓}
	\heartflow{願}
	\heartflow{學}
	%%%4
	\heartflow{佛}
	\heartflow{道}
	\heartflow{無}
	\heartflow{上}
	\heartflow{誓}
	\heartflow{願}
	\heartflow{成}
	\coda
	%%%
	%%% 1
	\heartflow{衆}
	\heartflow{生}
	\heartflow{無}
	\heartflow{邊}
	\heartflow{誓}
	\heartflow{願}
	\heartflow{度}
	%%% 2
	\heartflow{煩}
	\heartflow{悩}
	\heartflow{無}
	\heartflow{儘}
	\heartflow{誓}
	\heartflow{願}
	\heartflow{斷}
	%%% 3
	\heartflow{法}
	\heartflow{門}
	\heartflow{無}
	\heartflow{量}
	\heartflow{誓}
	\heartflow{願}
	\heartflow{學}
	%%%4
	\heartflow{佛}
	\heartflow{道}
	\heartflow{無}
	\heartflow{上}
	\heartflow{誓}
	\heartflow{願}
	\heartflow{成}
	\coda
\end{multicols}	

\begin{verbatim}

    \heartflow{}
    
\end{verbatim}
	
	\vfill\eject\pagebreak
	
	%%%%% SMALL FLOWING TEXT
	\vspace*{2cm}
	\renewcommand{\kanji}{\centering\fontsize{40}{40}}
	
	\textcolor{gray}{\kanji{四弘誓願}}
	\vspace*{2cm}
	
	\begin{multicols}{4}
		
		\renewcommand{\kanji}{\centering\fontsize{45}{45}}
		\RLmulticolcolumns
		%%% 1
		\heartspaced{衆}
		\heartspaced{生}
		\heartspaced{無}
		\heartspaced{邊}
		\heartspaced{誓}
		\heartspaced{願}
		\heartspaced{度}
		%%% 2
		\heartspaced{煩}
		\heartspaced{悩}
		\heartspaced{無}
		\heartspaced{儘}
		\heartspaced{誓}
		\heartspaced{願}
		\heartspaced{斷}
		%%% 3
		\heartspaced{法}
		\heartspaced{門}
		\heartspaced{無}
		\heartspaced{量}
		\heartspaced{誓}
		\heartspaced{願}
		\heartspaced{學}
		%%%4
		\heartspaced{佛}
		\heartspaced{道}
		\heartspaced{無}
		\heartspaced{上}
		\heartspaced{誓}
		\heartspaced{願}
		\heartspaced{成}
		
\end{multicols}

\begin{verbatim}
\renewcommand{\kanji}{\centering\fontsize{45}{45}}
\heartspaced{}
\end{verbatim}

\pagebreak

\vspace*{2cm}

\begin{multicols}{4}
	\RLmulticolcolumns
	\renewcommand{\kanji}{\centering\fontsize{35}{35}}
	\haiku{繰り返し麦の畝縫ふ胡蝶哉 \coda}
	\haiku{寒月に木を割る寺の男かな \coda}	
\end{multicols}

\begin{verbatim}
 
    \haiku{繰り返し麦の畝縫ふ胡蝶哉 \coda}
    
\end{verbatim}


\pagebreak

\vspace*{1cm}
\renewcommand{\kanji}{\centering\fontsize{25}{25}}
\begin{multicols}{7}
	\RLmulticolcolumns
	\spacedVerse{%
		般若心經ロ\stick マ字佛説摩訶般若波羅蜜多心經観自在菩薩行深般若波羅蜜多時照見五蘊皆空度 \stick 切苦厄
	}
\end{multicols}

\begin{verbatim}

	\spacedVerse{%
		般若心經ロ \stick マ...      %%% \stick for vertical 一
	}

\end{verbatim}

\pagebreak

\vspace*{2cm}

\renewcommand{\kanji}{\centering\fontsize{25}{25}}
\begin{multicols}{7}
	\RLmulticolcolumns
	\compactVerse{%
		般若心經ロ\stick マ字佛説摩訶般若波羅蜜多心經観自在菩薩行深般若波羅蜜多時照見五蘊皆空度 \stick 切苦厄
	}
\end{multicols}

\vspace*{1cm}

\begin{quotation}
	
	\begin{verbatim}
	
	\compactVerse{%
	   般若心經ロ \stick マ...     %%% \stick for vertical 一
	}
	
	\end{verbatim}
	
\end{quotation}


\end{document} 