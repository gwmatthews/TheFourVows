%    sutras for tracing, reading, chanting

\documentclass[letterpaper]{article}
\usepackage[margin=2cm]{geometry}

\usepackage{sutras}

%%% CHOICE OF FONTS
%\setCJKmainfont{KanjiStrokeOrders}  %% for tracing 
\setCJKmainfont{ipam.ttf}  %% serif
%\setCJKmainfont{ipaexm.ttf}  %% serif
%\setCJKmainfont{ipag.ttf}  %% sans
%\setCJKmainfont{ipamp.ttf}  %% sans
%\setCJKmainfont{ipaexg.ttf}  %% sans
%\setCJKmainfont{YOzM90.ttf}  %% calligraphic
%\setCJKmainfont{Harano Aji Mincho SemiBold}  %% calligraphic
%\setCJKmainfont{KouzanGyoushoOTF.otf}  %% calligraphic

%%%%%%%
%%%%%% END PREAMBLE
%%%%%%%

\begin{document}
	\vspace*{1cm}
	
	 {\centering\Huge 聖経 
		
		\vspace*{0.25cm}
		
		\href{https://github.com/gwmatthews/TheFourVows}{\texttt{sutras.sty}
		{\small for stylish sutras}}
		
		
	}

\vspace*{0.5cm}
	    \begin{quote}
	    	
	    	\begin{itemize}
	    		\item[{\fontsize{20}{20}{法}}] What is this anyway?
	    		\item[] First, it is a collection of printable versions of Japanese zen sutras for tracing, chanting, learning or display in a variety of styles and fonts. See next page for examples.
	    	\end{itemize}
	    	
	    	\begin{itemize}
	    		\item[{\fontsize{20}{20}{法}}] And for those into \href{https://www.latex-project.org/about/}{ \LaTeX, a document preparation system}
	    		\item[]  \textbf{Some \LaTeX code for typesetting kanji texts} -- built with Japanese in mind.
	    		\item[] Includes formats for tracing, reading (with or without meanings and readings), and display. From haiku to the heart sutra and beyond $\ldots$
	    		\item[] Examples are printable with different versions of the Four Bodhisattva Vows, The Heart Sutra, etc.
	    		\item[] Documentation to help you build your own throughout.
	    		\item[] \cc George Matthews \& Rose DiFonzo 2020
	    	\end{itemize}
	    \end{quote}
	    
	    
	    
\vspace*{1cm}
	
	\begin{quotation}
		
	\begin{verbatim}
	
	\usepackage{sutras}
	
	\begin{multicols}{4}
	\RLmulticolcolumns
	
	  \heartblock{r}{K}{m}               % reading above; KANJI large gray; meaning below
	
	  %%% invoked directly or implicitly by three commands below
	  
	  \heartflow{K}                      % dark gray text, variable boxes for KANJI 
	  \heartspaced{K}                    % variable boxes for KANJI with more padding
	 
	  
	  %%% for setting verse
	  \haiku{abc \coda}                  % two or more columns wide [ \coda = 。]
	  \compactVerse{abcdef...}           % tightly set verse
	  \spacedVerse{abcdef...}            % spaciously set verse
	
	\end{multicols}
	\end{verbatim}
	
\end{quotation}
	
	\vspace*{1cm}
 
	\begin{itemize}
		
			\item[] {\Large 
			 \href{https://github.com/gwmatthews/TheFourVows}{source code}}
			 \item[] {\Large 
			 	\href{https://gwmatthews.github.io/the-four-vows.pdf}{demo}}
		
	\end{itemize}
	\vfill\eject\pagebreak
	
	
	
	{\LARGE Flashcards}
	
	\vspace*{2cm}
	
	The flashcards/ subdirectory contains a linux bash script that automates the production of flashcards from csv data files. The data file format, which is the result of the wanikani Item Inspector script settings listed below looks like so:
	
	\begin{verbatim}
	
	"とう",刀,"Sword"
	"りょう",了,"Finish"
	"しょう",少,"Few"
	
	\end{verbatim}
	
	
	\begin{itemize}
		\item[] \LaTeX requirements: multicols, tcolorbox, xeCJK, anyfontsize, nopagno, xelatex, pdfpages (optional)
		\item[] Wanikani item inspector settings:
		\begin{itemize}
			\item[] filter: kanji and vocabulary only
			\item[] columns: Reading Brief, Item, Meaning Brief
			\item[] cell separator => comma, use of quotes => only when needed
		\end{itemize}
	\end{itemize}


\begin{verbatim}
$#> cp yourFile.csv flashcards/
$#> cd flashcards
$#> chmod +x flash.sh      # if needed
$#> ./flash.sh yourFile    # no .csv extension!

\end{verbatim}

What this script does is reformat the csv file, split it into sections that fit on a page, and combine these with the page templates \verb|cards.tex| and \verb|card-frame| to produce front and back pages in pdf format. Use a pdf reshuffler or the optional \verb|remix.tex| file for semi-automated reshuffling of pages so fronts and backs are together. I still have to figure out how to automate the reshuffling of the card sheets.	
	
	
	\vfill\eject\pagebreak
	
	\vspace*{2cm}
	
	{\LARGE Examples}
	
	\vspace*{1cm}
	
	\begin{itemize}
		\item[] \href{https://gwmatthews.github.io/the-four-vows.pdf}{四弘誓願: the four bodhisattva vows}
		\item[] \href{https://gwmatthews.github.io/the-four-vows-plain.pdf}{四弘誓願: the four bodhisattva vows, plain version}
		\item[] \href{https://gwmatthews.github.io/the-heart-sutra.pdf}{般若心経: the heart sutra}
		\item[] \href{https://gwmatthews.github.io/kanzeon.pdf}{延命十句觀音經: the ten verse kannon sutra}
		\item[] \href{https://gwmatthews.github.io/leeches-cards.pdf}{flashcards}
		\item[] \href{https://gwmatthews.github.io/examples/stroke-order.pdf}{KanjiStrokeOrders.ttf}
		\item[] \href{https://gwmatthews.github.io/examples/ipaexm.pdf}{ipaexm.ttf}
		\item[] \href{https://gwmatthews.github.io/examples/ipamp.pdf}{ipamp.ttf}
		\item[] 
			\href{https://gwmatthews.github.io/examples/ipam.pdf}{ipam.ttf}
		\item[] \href{https://gwmatthews.github.io/examples/ipaexg.pdf}{ipaexg.ttf}
		\item[] 
			\href{https://gwmatthews.github.io/examples/ipag.pdf}{ipag.ttf}
		\item[] \href{https://gwmatthews.github.io/examples/HaranoAjiMinchoSB.pdf}{Harano Aji Mincho SemiBold}
		\item[] \href{https://gwmatthews.github.io/examples/YOzM90.pdf}{YOzM90.ttf}
		\item[] \href{https://gwmatthews.github.io/examples/KouzanGyousho.pdf}{KouzanGyoushoOTF.otf}
		\item[] \href{https://gwmatthews.github.io/examples/verse-test.pdf}{verse styles}
		
\end{itemize}

\vfill\eject\pagebreak
\newgeometry{margin=0.5cm}
\centering 四弘誓願 The Four Bodhisattva Vows

\begin{multicols}{4}
\RLmulticolcolumns
\renewcommand{\kanji}{\centering\fontsize{45}{45}}

\heartblock{しゅう shu}{衆}{many}
\heartblock{じょう jou}{生}{life \\ 衆生 all living things}
\heartblock{む mu}{無}{nothing}
\heartblock{へん hen}{邊}{boundary \\ 邊 is now 辺,  無辺 limitless}
\heartblock{せい sei}{誓}{vow}
\heartblock{がん gan}{願}{vow  \\ 誓願 vow}
\heartblock{ど do}{度}{degrees \\ paramita, save}

\heartblock{ぽん hon}{煩}{anxiety}
\heartblock{のう nou}{悩}{trouble \\ 煩悩 three poisons }
\heartblock{む mu}{無}{no}
\heartblock{じん jin}{儘}{exhaust \\ simplifies to 盡,  侭,  尽}
\heartblock{せい sei}{誓}{}
\heartblock{がん gan}{願}{}
\heartblock{だん dan}{斷}{quit \\ simplifies to 断}

\heartblock{ほう hou}{法}{dharma}
\heartblock{もん mon}{門}{gate}
\heartblock{む mu}{無}{no}
\heartblock{りょう ryo}{量}{quantity \\ 無量 immeasurable}
\heartblock{せい sei}{誓}{}
\heartblock{がん gan}{願}{}
\heartblock{がく gaku}{學}{learn \\ simplifies to 学}

\heartblock{ぶつ butsu}{佛}{Buddha}
\heartblock{どう dou}{道}{path}
\heartblock{む mu}{無}{no}
\heartblock{じょう jou}{上}{above \\ 無上 best}
\heartblock{せい sei}{誓}{}
\heartblock{がん gan}{願}{}
\heartblock{じょう jou}{成}{become}

\end{multicols}

\begin{verbatim}

    \heartblock{}{}{}

\end{verbatim}

\pagebreak
\newgeometry{margin=0.3cm}
\renewcommand{\kanji}{\centering\fontsize{60}{60}}

\vspace*{0.5cm}
\begin{multicols}{4}
	\RLmulticolcolumns
	\heartblock{}{衆}{}
	\heartblock{}{生}{}
	\heartblock{}{無}{}
	\heartblock{}{邊}{}
	\heartblock{}{誓}{}
	\heartblock{}{願}{}
	\heartblock{}{度}{}
	
	\columnbreak
	
	\heartblock{}{煩}{}
	\heartblock{}{悩}{}
	\heartblock{}{無}{}
	\heartblock{}{儘}{}
	\heartblock{}{誓}{}
	\heartblock{}{願}{}
	\heartblock{}{斷}{}
	
	\columnbreak
	
	\heartblock{}{法}{}
	\heartblock{}{門}{}
	\heartblock{}{無}{}
	\heartblock{}{量}{}
	\heartblock{}{誓}{}
	\heartblock{}{願}{}
	\heartblock{}{學}{}
	
	\columnbreak
	
	\heartblock{}{佛}{}
	\heartblock{}{道}{}
	\heartblock{}{無}{}
	\heartblock{}{上}{}
	\heartblock{}{誓}{}
	\heartblock{}{願}{}
	\heartblock{}{成}{}
	
\end{multicols}

\begin{verbatim}

    \heartblock{}{無}{}
    
\end{verbatim}

\pagebreak

\newgeometry{margin=0.5cm}

%%%%% MEDIUM FLOWING TEXT

\vspace*{1cm}

\renewcommand{\kanji}{\centering\fontsize{55}{55}}
\vspace*{2cm}

\begin{multicols}{4}
	\RLmulticolcolumns
	
	\heartflow{衆}
	\heartflow{生}
	\heartflow{無}
	\heartflow{邊}
	\heartflow{誓}
	\heartflow{願}
	\heartflow{度}
	\columnbreak
		
	\heartflow{煩}
	\heartflow{悩}
	\heartflow{無}
	\heartflow{儘}
	\heartflow{誓}
	\heartflow{願}
	\heartflow{斷}
	\columnbreak

	\heartflow{法}
	\heartflow{門}
	\heartflow{無}
	\heartflow{量}
	\heartflow{誓}
	\heartflow{願}
	\heartflow{學}
	\columnbreak
	
	\heartflow{佛}
	\heartflow{道}
	\heartflow{無}
	\heartflow{上}
	\heartflow{誓}
	\heartflow{願}
	\heartflow{成}

\end{multicols}

\begin{verbatim}

    \heartflow{}
    
\end{verbatim}

\pagebreak

%%%%% SMALL FLOWING TEXT
\vspace*{2cm}
\renewcommand{\kanji}{\centering\fontsize{40}{40}}
\textcolor{gray}{\kanji{四弘誓願}}
\vspace*{2cm}

\renewcommand{\kanji}{\centering\fontsize{35}{35}}
\begin{multicols}{7}
	
	\RLmulticolcolumns
	%%% 1
	\heartflow{衆}
	\heartflow{生}
	\heartflow{無}
	\heartflow{邊}
	\heartflow{誓}
	\heartflow{願}
	\heartflow{度}
	%%% 2
	\heartflow{煩}
	\heartflow{悩}
	\heartflow{無}
	\heartflow{儘}
	\heartflow{誓}
	\heartflow{願}
	\heartflow{斷}
	%%% 3
	\heartflow{法}
	\heartflow{門}
	\heartflow{無}
	\heartflow{量}
	\heartflow{誓}
	\heartflow{願}
	\heartflow{學}
	%%%4
	\heartflow{佛}
	\heartflow{道}
	\heartflow{無}
	\heartflow{上}
	\heartflow{誓}
	\heartflow{願}
	\heartflow{成}
    \coda
	%%%
	%%% 1
	\heartflow{衆}
	\heartflow{生}
	\heartflow{無}
	\heartflow{邊}
	\heartflow{誓}
	\heartflow{願}
	\heartflow{度}
	%%% 2
	\heartflow{煩}
	\heartflow{悩}
	\heartflow{無}
	\heartflow{儘}
	\heartflow{誓}
	\heartflow{願}
	\heartflow{斷}
	%%% 3
	\heartflow{法}
	\heartflow{門}
	\heartflow{無}
	\heartflow{量}
	\heartflow{誓}
	\heartflow{願}
	\heartflow{學}
	%%%4
	\heartflow{佛}
	\heartflow{道}
	\heartflow{無}
	\heartflow{上}
	\heartflow{誓}
	\heartflow{願}
	\heartflow{成}
	\coda
	%%%
	%%% 1
	\heartflow{衆}
	\heartflow{生}
	\heartflow{無}
	\heartflow{邊}
	\heartflow{誓}
	\heartflow{願}
	\heartflow{度}
	%%% 2
	\heartflow{煩}
	\heartflow{悩}
	\heartflow{無}
	\heartflow{儘}
	\heartflow{誓}
	\heartflow{願}
	\heartflow{斷}
	%%% 3
	\heartflow{法}
	\heartflow{門}
	\heartflow{無}
	\heartflow{量}
	\heartflow{誓}
	\heartflow{願}
	\heartflow{學}
	%%%4
	\heartflow{佛}
	\heartflow{道}
	\heartflow{無}
	\heartflow{上}
	\heartflow{誓}
	\heartflow{願}
	\heartflow{成}
	\coda
\end{multicols}	

\begin{verbatim}

    \heartflow{}
    
\end{verbatim}
	
	\pagebreak
	
	%%%%% SMALL FLOWING TEXT
	\vspace*{2cm}
	\renewcommand{\kanji}{\centering\fontsize{40}{40}}
	
	\textcolor{gray}{\kanji{四弘誓願}}
	\vspace*{2cm}
	
	\begin{multicols}{4}
		
		\renewcommand{\kanji}{\centering\fontsize{45}{45}}
		\RLmulticolcolumns
		%%% 1
		\heartspaced{衆}
		\heartspaced{生}
		\heartspaced{無}
		\heartspaced{邊}
		\heartspaced{誓}
		\heartspaced{願}
		\heartspaced{度}
		%%% 2
		\heartspaced{煩}
		\heartspaced{悩}
		\heartspaced{無}
		\heartspaced{儘}
		\heartspaced{誓}
		\heartspaced{願}
		\heartspaced{斷}
		%%% 3
		\heartspaced{法}
		\heartspaced{門}
		\heartspaced{無}
		\heartspaced{量}
		\heartspaced{誓}
		\heartspaced{願}
		\heartspaced{學}
		%%%4
		\heartspaced{佛}
		\heartspaced{道}
		\heartspaced{無}
		\heartspaced{上}
		\heartspaced{誓}
		\heartspaced{願}
		\heartspaced{成}
		
\end{multicols}

\begin{verbatim}
\renewcommand{\kanji}{\centering\fontsize{45}{45}}
\heartspaced{}
\end{verbatim}

\pagebreak

\vspace*{2cm}

\begin{multicols}{4}
	\RLmulticolcolumns
	\renewcommand{\kanji}{\centering\fontsize{35}{35}}
	\haiku{繰り返し麦の畝縫ふ胡蝶哉 \coda}
	\haiku{寒月に木を割る寺の男かな \coda}	
\end{multicols}

\begin{verbatim}
 
    \haiku{繰り返し麦の畝縫ふ胡蝶哉 \coda}
    
\end{verbatim}


\pagebreak

\vspace*{1cm}
\renewcommand{\kanji}{\centering\fontsize{25}{25}}
\begin{multicols}{7}
	\RLmulticolcolumns
	\spacedVerse{%
		般若心經ロ一マ字佛説摩訶般若波羅蜜多心經観自在菩薩行深般若波羅蜜多時照見五蘊皆空度一切苦厄
	}
\end{multicols}

\begin{verbatim}

	\spacedVerse{%
		般若心經ロ一マ...
	}

\end{verbatim}

\pagebreak

\vspace*{2cm}

\renewcommand{\kanji}{\centering\fontsize{25}{25}}
\begin{multicols}{7}
	\RLmulticolcolumns
	\compactVerse{%
		般若心經ロ一マ字佛説摩訶般若波羅蜜多心經観自在菩薩行深般若波羅蜜多時照見五蘊皆空度一切苦厄
	}
\end{multicols}

\vspace*{1cm}

\begin{quotation}
	
	\begin{verbatim}
	
	\compactVerse{%
	   般若心經ロ一マ... 
	}
	
	\end{verbatim}
	
\end{quotation}

\shoutout

\end{document} 
