%    printable Heart Sutra for Tracing

\documentclass[letterpaper]{article}
\usepackage[left=2cm, right=2cm, top=0.5cm, bottom=0.5cm]{geometry}
\usepackage{nopageno}
\usepackage{xeCJK}
\usepackage{anyfontsize}
\usepackage{hyperref}
\usepackage[most]{tcolorbox}

\hypersetup {
 pdfauthor= Rose DiFonzo (remix: George Matthews),
 pdftitle={Four Bodhisattva Vows for Tracing},
 pdfsubject={Buddhism, kanji, 四弘誓願  },
 pdfkeywords={kanji, japan, wanikani, Bodhisattva, Buddhism, Buddha, 四弘誓願 },
 }
 
\setCJKmainfont{KanjiStrokeOrders}
\setCJKsansfont{Meiryo}
  
\newcommand{\kanji}{\tcbitem\centering\fontsize{55}{55}}
\newcommand{\largekanji}{\tcbitem\centering\fontsize{75}{75}}
\newcommand{\reading}{\tcbitem\centering\fontsize{10}{10}}
\newcommand{\means}{\tcbitem\centering\fontsize{8}{8}}

\begin{document}
	
\centering 四弘誓願 The Four Bodhisattva Vows

\tcbset{colframe=white, colback=white, raster valign=center, raster halign=center, raster columns=4, size=minimal}

%%%%%%%%%%% ROW 1

\vspace*{1cm}
	
\begin{tcbitemize}
	\reading ぶつ 
	\reading ほう 
	\reading ぽん 
	\reading しゅ 
\end{tcbitemize}

\begin{tcbitemize}[coltext=lightgray]
	\kanji 衆
	\kanji 煩
	\kanji 法
	\kanji 佛
\end{tcbitemize}

\begin{tcbitemize}
	\means Buddha
	\means dharma
	\means anxiety
	\means many
\end{tcbitemize}

%%%%%%%%%%% ROW 2
\vspace*{3mm}


\begin{tcbitemize}
	\reading どう 
	\reading もん 
	\reading のう 
	\reading じょう 
\end{tcbitemize}

\begin{tcbitemize}[coltext=lightgray]
	\kanji 道
	\kanji 門
	\kanji 悩
	\kanji 生
\end{tcbitemize}

\begin{tcbitemize}
	\means path
	\means gate
	\means trouble \\ 煩悩 three poisons 
	\means life \\ 衆生 all living things
\end{tcbitemize}

%%%%%%%%%%% ROW 3
\vspace*{3mm}

\begin{tcbitemize}
	\reading む 
	\reading む 
	\reading む 
	\reading む 
\end{tcbitemize}

\begin{tcbitemize}[coltext=lightgray]
	\kanji 無
	\kanji 無
	\kanji 無
	\kanji 無
\end{tcbitemize}

\begin{tcbitemize}
	\means no
	\means no
	\means no 
	\means no 
\end{tcbitemize}


%%%%%%%%%%% ROW 4
\vspace*{3mm}

\begin{tcbitemize}
	\reading じょう 
	\reading りょう 
	\reading じん 
	\reading へん 
\end{tcbitemize}

\begin{tcbitemize}[coltext=lightgray]
	\kanji 上
	\kanji 量
	\kanji 儘
	\kanji 邊
\end{tcbitemize}

\begin{tcbitemize}
	\means above \\ 無上 best
	\means quantity \\ 無量 immeasurable
	\means exhaust \\ simplifies to 盡,  侭,  尽 
	\means boundary \\ 邊 is now 辺 -- 無辺 limitless 
\end{tcbitemize}


%%%%%%%%%%% ROW 5
\vspace*{3mm}

\begin{tcbitemize}
	\reading せい 
	\reading せい 
	\reading せい 
	\reading せい 
\end{tcbitemize}


\begin{tcbitemize}[coltext=lightgray]
	\kanji 誓
	\kanji 誓
	\kanji 誓
	\kanji 誓
\end{tcbitemize}

\begin{tcbitemize}
	\means vow
	\means vow
	\means vow 
	\means vow 
\end{tcbitemize}


%%%%%%%%%%% ROW 6
\vspace*{3mm}

\begin{tcbitemize}
	\reading がん 
	\reading がん 
	\reading がん
	\reading がん 
\end{tcbitemize}

\begin{tcbitemize}[coltext=lightgray]
	\kanji 願
	\kanji 願
	\kanji 願
	\kanji 願
\end{tcbitemize}

\begin{tcbitemize}
	\means 誓願 vow
	\means 誓願 vow
	\means 誓願 vow
	\means 誓願 vow 
\end{tcbitemize}


%%%%%%%%%%% ROW 7
\vspace*{3mm}

\begin{tcbitemize}
	\reading じょう
	\reading がく
	\reading だん
	\reading ど
\end{tcbitemize}

\begin{tcbitemize}[coltext=lightgray]
	\kanji 成
	\kanji 學
	\kanji 斷
	\kanji 度
\end{tcbitemize}

\begin{tcbitemize}
	\means become
	\means learn \\ simplifies to 学
	\means quit \\ simplifies to 断
	\means degrees, paramita, save 
\end{tcbitemize}


\pagebreak

%%%%%%% LARGE KANJI, NO READINGS OR MEANINGS

\vspace*{2cm}

\begin{tcbitemize}[coltext=lightgray]
	\largekanji 衆
	\largekanji 煩
	\largekanji 法
	\largekanji 佛
\end{tcbitemize}

\vspace*{1cm}

\begin{tcbitemize}[coltext=lightgray]
	\largekanji 道
	\largekanji 門
	\largekanji 悩
	\largekanji 生
\end{tcbitemize}

\vspace*{1cm}

\begin{tcbitemize}[coltext=lightgray]
	\largekanji 無
	\largekanji 無
	\largekanji 無
	\largekanji 無
\end{tcbitemize}

\vspace*{1cm}

\begin{tcbitemize}[coltext=lightgray]
	\largekanji 上
	\largekanji 量
	\largekanji 儘
	\largekanji 邊
\end{tcbitemize}

\vspace*{1cm}

\begin{tcbitemize}[coltext=lightgray]
	\largekanji 誓
	\largekanji 誓
	\largekanji 誓
	\largekanji 誓
\end{tcbitemize}

\vspace*{1cm}

\begin{tcbitemize}[coltext=lightgray]
	\largekanji 成
	\largekanji 學
	\largekanji 斷
	\largekanji 度
\end{tcbitemize}

\end{document} 