%    sutras for tracing, reading, chanting

\documentclass[letterpaper]{article}
\usepackage[margin=2cm]{geometry}

\usepackage{sutras}

%%% CHOICE OF FONTS

%\setCJKmainfont{KanjiStrokeOrders}  %% for tracing 
%\setCJKmainfont{ipam.ttf}  %% serif
\setCJKmainfont{ipaexm.ttf}  %% serif
%\setCJKmainfont{ipag.ttf}  %% sans
%\setCJKmainfont{ipamp.ttf}  %% sans
%\setCJKmainfont{ipaexg.ttf}  %% sans
%\setCJKmainfont{YOzM90.ttf}  %% calligraphic
%\setCJKmainfont{Harano Aji Mincho SemiBold}  %% calligraphic
%\setCJKmainfont{KouzanGyoushoOTF.otf}  %% calligraphic

%%%%%%%
%%%%%% END PREAMBLE
%%%%%%%



\begin{document}
	
		\vspace*{2cm}
	
	\renewcommand{\kanji}{\centering\fontsize{55}{55}}
	\begin{multicols}{3}
		\RLmulticolcolumns
		
		\haiku{古池や蛙飛び込む水の音 \coda {\fontsize{30}{30} 芭蕉 }}
		
	\end{multicols}
	
	\vspace*{1cm}
	
	\begin{verbatim}
	\renewcommand{\kanji}{\centering\fontsize{55}{55}}
	
	\begin{multicols}{3}
	\RLmulticolcolumns
	
	   \haiku{古池や蛙飛び込む水の音 \coda {\fontsize{30}{30} 芭蕉 }}
	
	\end{multicols}
	
	\end{verbatim}
	
	\pagebreak
	
	\vspace*{2cm}
			
	\renewcommand{\kanji}{\centering\fontsize{35}{35}}
	\begin{multicols}{4}
		\RLmulticolcolumns\raggedcolumns
		
		\haiku{古池や}
		\columnbreak
		\haiku{蛙飛び込む}
		\columnbreak
		\haiku{水の音}
		\columnbreak
		\haiku{\coda{ 芭蕉 }\coda}
			
	\end{multicols}

\vspace*{1cm}

\begin{verbatim}
\renewcommand{\kanji}{\centering\fontsize{35}{35}}

\begin{multicols}{4}
\RLmulticolcolumns\raggedcolumns

  \haiku{古池や}
  \columnbreak

  \haiku{蛙飛び込む}
  \columnbreak

  \haiku{水の音}
  \columnbreak

  \haiku{\coda{ 芭蕉 }\coda}

\end{multicols}

\end{verbatim}
	
\pagebreak


\newgeometry{margin=2.5cm}
	
	\renewcommand{\kanji}{\centering\fontsize{30}{30}}
	\begin{multicols}{7}
		\RLmulticolcolumns
		\spacedVerse{%
			般若心經ロ\stick マ字佛説摩訶般若波羅蜜多心經観自在菩薩行深般若波羅蜜多時照見五蘊皆空度 \stick 切苦厄舎利子色不異空空不異色色即是空空即是色受想行識亦復如是舎利子是諸法空相不生不滅不垢不浄不増不減是故空中無色無受想行識無眼耳鼻舌身意無色聲香味触法無眼界乃至無意識界無無明亦無無明盡乃至無老死亦無老死盡無苦集滅道無智亦無得以無所得故菩提薩埵依般若波羅蜜多故心無罣礙無罣礙故無有恐怖遠離\stick 切顛倒夢想究竟涅槃三世諸佛。依般若波羅蜜多故得阿耨多羅三藐三菩提故知般若波羅蜜多是大神咒是大明咒是無上咒是無等等咒能除\stick 切苦真實不虚故説般若波羅蜜多咒即説咒曰羯諦羯諦波羅羯諦波羅僧羯諦菩提薩婆訶般若心經}
	\end{multicols}
	
	
\begin{verbatim}

\renewcommand{\kanji}{\centering\fontsize{30}{30}}

\begin{multicols}{7}
  \RLmulticolcolumns
      
    \spacedVerse{
      般若心經ロ\stickマ字佛説摩...
    }
  
\end{multicols}
	
\end{verbatim}
	
	\pagebreak
	
	
	\renewcommand{\kanji}{\centering\fontsize{34}{34}}
	\begin{multicols}{7}
		\RLmulticolcolumns
		\compactVerse{般若心經ロ\stick マ字佛説摩訶般若波羅蜜多心經観自在菩薩行深般若波羅蜜多時照見五蘊皆空度\stick 切苦厄舎利子色不異空空不異色色即是空空即是色受想行識亦復如是舎利子是諸法空相不生不滅不垢不浄不増不減是故空中無色無受想行識無眼耳鼻舌身意無色聲香味触法無眼界乃至無意識界無無明亦無無明盡乃至無老死亦無老死盡無苦集滅道無智亦無得以無所得故菩提薩埵依般若波羅蜜多故心無罣礙無罣礙故無有恐怖遠離\stick 切顛倒夢想究竟涅槃三世諸佛。依般若波羅蜜多故得阿耨多羅三藐三菩提故知般若波羅蜜多是大神咒是大明咒是無上咒是無等等咒能除\stick 切苦真實不虚故説般若波羅蜜多咒即説咒曰羯諦羯諦波羅羯諦波羅僧羯諦菩提薩婆訶般若心經}
	\end{multicols}
	
	
\begin{verbatim}

\renewcommand{\kanji}{\centering\fontsize{34}{34}}

\begin{multicols}{7}
  \RLmulticolcolumns

    \compactVerse{
      般若心經ロ\stickマ字佛説摩...
    }

\end{multicols}
	
\end{verbatim}
	
\end{document} 