%    sutras for tracing, reading, chanting

\documentclass[letterpaper]{article}
\usepackage[margin=2cm]{geometry}



%%% Provides \heartblock{}{}{}   \heartflow{}  \heartspaced{}


%%% NEW COMMANDS
\newcommand{\kanji}{\centering\fontsize{60}{60}}   %%% set main font size 
\newcommand{\means}{\tcblower\centering}           %%% position lower text

%%% For ending verses
\newcommand{\coda}{	\vspace*{1ex}\heartflow{。}\vspace*{1ex}}


%%%%%%%%%%%%%%
% FIXED HEIGHT BOX FOR INCLUDING READINGS AND/OR MEANINGS
% #1: tcolorbox options
% #2: box title

\newtcolorbox{kanjibox}[2][]
{
	colframe = black!0,
	height=3.3cm,    %%%% fixed height boxes!
	size=small,
	colback  = white,
	colupper= black!30,
	coltitle = black,
	halign title= center,
	fontlower= \scriptsize,
	fonttitle= \small, 
	title    = {#2},
	#1,
}

%%% display reading, kanji and meaning
\newcommand{\heartblock}[3]
{
	\begin{kanjibox}{#1}
		\kanji #2
		\means #3
	\end{kanjibox}
}
%%%%%%%%%%%%%%%%



%%%%%%%%%%%%%%%%
% VARIABLE HEIGHT BLOCK FOR JUST KANJI
% #1: tcolorbox options

\newtcolorbox{kanjiflow}[1][]
{
	colframe = black!0,
	size=tight,
	colback  = white,
	colupper= black!75,
	coltitle = black,
	halign title= center,
	fontlower= \tiny,
	fonttitle= \tiny,
	#1,
}

%%% display kanji

\newcommand{\heartflow}[1]
{
	\begin{kanjiflow}[size=tight]{}
		\kanji #1
	\end{kanjiflow}
}


%%% display kanji with more padding

\newcommand{\heartspaced}[1]
{
	\begin{kanjiflow}[size=normal, boxsep=1ex]{}
		\kanji #1
	\end{kanjiflow}
}

\endinput
%%% CHOICE OF FONTS

%\setCJKmainfont{KanjiStrokeOrders}  %% for tracing 
%\setCJKmainfont{ipam.ttf}  %% serif
%\setCJKmainfont{ipaexm.ttf}  %% serif
%\setCJKmainfont{ipag.ttf}  %% sans
%\setCJKmainfont{ipamp.ttf}  %% sans
%\setCJKmainfont{ipaexg.ttf}  %% sans
\setCJKmainfont{YOzM90.ttf}  %% calligraphic
%\setCJKmainfont{Harano Aji Mincho SemiBold}  %% calligraphic
%\setCJKmainfont{KouzanGyoushoOTF.otf}  %% calligraphic

%%%%%%%
%%%%%% END PREAMBLE
%%%%%%%

\begin{document}
	\vspace*{2cm}
		
	\begin{multicols}{4}
		\RLmulticolcolumns
		\renewcommand{\kanji}{\centering\fontsize{35}{35}}
		\haiku{繰,り,返,し,麦,の,畝,縫,ふ,胡,蝶,哉}
		\haiku{寒,月,に,木,を,割,る,寺,の,男,か,な}	
	\end{multicols}

\begin{verbatim}
\begin{multicols}{4}
\RLmulticolcolumns
\renewcommand{\kanji}{\centering\fontsize{35}{35}}
    \haiku{繰,り,返,し,麦,の,畝,縫,ふ,胡,蝶,哉}
    \haiku{寒,月,に,木,を,割,る,寺,の,男,か,な}	
\end{multicols}
\end{verbatim}
	
	
	\pagebreak
	\newgeometry{margin=2cm}
	
	\renewcommand{\kanji}{\centering\fontsize{30}{30}}
	\begin{multicols}{7}
		\RLmulticolcolumns
		\spacedverse{
	般,若,心,經,ロ,\stick,マ,字,佛,説,摩,訶,般,若,波,羅,蜜,多,心,經,観,自,在,菩,薩,行,深,般,若,波,羅,蜜,多,時,照,見,五,蘊,皆,空,度,\stick,切,苦,厄,舎,利,子,色,不,異,空,空,不,異,色,色,即,是,空,空,即,是,色,受,想,行,識,亦,復,如,是, ,舎,利,子,是,諸,法,空,相,不,生,不,滅,不,垢,不,浄,不,増,不,減,是,故,空,中,無,色,無,受,想,行,識,無,眼,耳,鼻,舌,身,意,無,色,聲,香,味,触,法,無,眼,界,乃,至,無,意,識,界,無,無,明,亦,無,無,明,盡,乃,至,無,老,死,亦,無,老,死,盡,無,苦,集,滅,道,無,智,亦,無,得,以,無,所,得,故,菩,提,薩,埵,依,般,若,波,羅,蜜,多,故,心,無,罣,礙,無,罣,礙,故,無,有,恐,怖,遠,離,\stick,切,顛,倒,夢,想,究,竟,涅,槃,三,世,諸,佛,。,依,般,若,波,羅,蜜,多,故,得,阿,耨,多,羅,三,藐,三,菩,提,故,知,般,若,波,羅,蜜,多,是,大,神,咒,是,大,明,咒,是,無,上,咒,是,無,等,等,咒,能,除,\stick,切,苦,真,實,不,虚,故,説,般,若,波,羅,蜜,多,咒,即,説,咒,曰,羯,諦,羯,諦,波,羅,羯,諦,波,羅,僧,羯,諦,菩,提,薩,婆,訶,般,若,心,經
	}
	\end{multicols}
	
	
	\begin{verbatim}
	\renewcommand{\kanji}{\centering\fontsize{30}{30}}
	\begin{multicols}{7}
	\RLmulticolcolumns
		\spacedverse{
			般,若,心,經,ロ,\stick,マ,...
		}
	\end{multicols}
	\end{verbatim}
	
	\pagebreak
	
	
	\renewcommand{\kanji}{\centering\fontsize{30}{30}}
	\begin{multicols}{7}
		\RLmulticolcolumns
		\compactverse{
			般,若,心,經,ロ,ー,マ,字,佛,説,摩,訶,般,若,波,羅,蜜,多,心,經,観,自,在,菩,薩,行,深,般,若,波,羅,蜜,多,時,照,見,五,蘊,皆,空,度,\stick,切,苦,厄,舎,利,子,色,不,異,空,空,不,異,色,色,即,是,空,空,即,是,色,受,想,行,識,亦,復,如,是, ,舎,利,子,是,諸,法,空,相,不,生,不,滅,不,垢,不,浄,不,増,不,減,是,故,空,中,無,色,無,受,想,行,識,無,眼,耳,鼻,舌,身,意,無,色,聲,香,味,触,法,無,眼,界,乃,至,無,意,識,界,無,無,明,亦,無,無,明,盡,乃,至,無,老,死,亦,無,老,死,盡,無,苦,集,滅,道,無,智,亦,無,得,以,無,所,得,故,菩,提,薩,埵,依,般,若,波,羅,蜜,多,故,心,無,罣,礙,無,罣,礙,故,無,有,恐,怖,遠,離,\stick,切,顛,倒,夢,想,究,竟,涅,槃,三,世,諸,佛,。,依,般,若,波,羅,蜜,多,故,得,阿,耨,多,羅,三,藐,三,菩,提,故,知,般,若,波,羅,蜜,多,是,大,神,咒,是,大,明,咒,是,無,上,咒,是,無,等,等,咒,能,除,\stick,切,苦,真,實,不,虚,故,説,般,若,波,羅,蜜,多,咒,即,説,咒,曰,羯,諦,羯,諦,波,羅,羯,諦,波,羅,僧,羯,諦,菩,提,薩,婆,訶,般,若,心,經
		}
	\end{multicols}
	
	
	\begin{verbatim}
	\renewcommand{\kanji}{\centering\fontsize{30}{30}}
	\begin{multicols}{7}
	\RLmulticolcolumns
		\compactverse{
			般,若,心,經,ロ,\stick,マ,...
	\sqrt{\sqrt{}}	}
	\end{multicols}
	\end{verbatim}
	
\end{document} 